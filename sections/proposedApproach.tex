There are several domain-specific MDE approaches and tools that can be reused for developing IoT-based architectures and applications. The IoT-based smart system architectures are very large software platforms which can be developing by different developers. Thus, for the sake of simplicity and to achieve effective results, the large-scale smart system architectures are strictly divided into parts and developed by several developers. After their initial deployment, they further have to be maintained, consisted, evolved, yet remain flexible and sustainable. These efficient results can be achieved using MDE trends and collaborative development support by sharing the artifacts of smart system architectures and models. 

The most existing MDE approaches and tools use standard UML profiles and meta-models as their underlying modeling concepts. However, support for collaborative MDE for IoT is still not well developed. A meta-model generic, textual difference representation language for collaborative modeling was introduced in \cite{Kuryazov+2018}. As long as the approach is meta-model generic, the same approach can be applied to collaborative modeling for the IoT-based smart system architectures. The approach will be applied to the existing tools using their underlying meta-models.

A meta-model and tool generic as well as flexible and sustainable \cite{jelschen+2016} collaborative MDE infrastructure introduced in \cite{Kuryazov+2018} consists of a three-layer architecture:
\begin{itemize}
\item[--] \textbf{Languages.} \cite{Kuryazov+2018} proposes a difference language (DL) for representing model changes in modeling deltas and develop the collaborative modeling approaches on the top. Model changes in modeling deltas are represented by a textual, operation-based DL which is a family of domain-specific languages. Specific DLs for domain-specific modeling languages are then generated by a DL generator service, importing the meta-models of modeling languages. The most existing MDE tools for the IoT-based smart systems are developed using standard UML profiles. In the framework of this proposal, a new DL will be extended and generated to apply to the model-driven IoT systems.
\item[--] \textbf{Services.} DL further supports a catalog of flexible and sustainable services for extending its application areas. In order to apply DL to model-driven IoT systems and collaborative modeling approach for IoT-based smart systems, supported list of DL services will be adapted and extended accordingly.
\item[--] \textbf{Applications.} After generating required DL and adapting its services, collaborative modeling use case will be realized by specific orchestrations of adapted DL services. Like in \cite{Kuryazov+2018}, this proposal also suggests two main scenarios: (1) interactivity of collaborators; (2) consistency of centralized artifact repositories. The collaborative modeling approach in \cite{Kuryazov+2018} introduces two main scenarios of collaborative MDE: micro-versioning and macro-versioning. Micro-versioning supports the interactivity of several developers on developing model-driven IoT architectures and models. Macro-versioning supports the consistency of model-driven IoT architectures.
\end{itemize}
