The IoT-based smart system architectures are very large software platforms and architectures. Thus, for the sake of simplicity and to achieve effective results, the large-scale smart system architectures can be developed using MDE trends. After their initial deployment, they further have to be maintained, consisted, evolved, yet remain flexible and sustainable. These efficient results can also be achieved using MDE trends. 

In order to achieve efficient research results in model-driven IoT, research should cover several steps:
\begin{itemize}
\item[--] \textbf{Meta-Model.} UML proposes a general modeling concepts for developing model-driven approaches. As long as UML can not directly applied to the IoT domain, the standard concepts of UML have to be extended and adapted according to the requirements of that domain. In the initial phase, this research proposes to develop a meta-model for developing model-driven IoT approach extending the standard UML profiles. This will be achieved by developing an adapter which can automatically extend standard UML profiles according to certain adaptation rules.
\item[--] \textbf{Tool.} This research further focuses on developing general model-driven application dedicated especially to IoT systems. It aims to support flexible and sustainable services such as model designing editor, model repository, etc. This application will be developed using the meta-model developed in the initial step.
\end{itemize}
