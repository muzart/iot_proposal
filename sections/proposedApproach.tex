There are several domain-specific MDE approaches and tools that can be reused for developing IoT-based architectures and applications. For the sake of simplicity, large-scale smart system architectures are strictly divided into parts and developed by several developers. maintenance and consistency, evolution and flexibility, they mostly lack collaborative designing and development support by sharing the artifacts of smart city architectures, models and applications among collaborators. The most existing tools use standard UML profiles by defining their own notations. However, these tools are developed on the top of EMF – eclipse modeling framework, Sirius and Obeo Designer using Ecore-based meta-modeling feature and standard UML meta-models. The collaborative modeling infrastructure introduced in Section 3 will be applied to these existing tools using their underlying meta-models.

A meta-model and modeling tool generic as well as flexible collaborative MDE infrastructure was introduced in [4]. The overall approach consists of a three-layer architecture namely: language generation, service orchestration and applications. In order to use the collaborative MDE infrastructure in any modeling domain, a difference language has to be generated for each domain. After generating difference languages, other underlying services can be orchestrated for building collaborative MDE applications for each domain language. These tech­niques are also applicable to the collaborative MDE of smart city applications.
Language Generation. The collaborative modeling approach takes advantage of modeling deltas [4] as difference documents for storing and exchanging model changes among collaborators. Model changes in modeling deltas are represented by a textual, operation-based Difference Language (DL). Formally, DL is a family of domain-specific languages. Specific DLs for domain-specific modeling languages can be generated by DL Generator, importing the meta-models of modeling languages. For instance, the approach is applied to UML class diagrams by importing the UML class diagram meta-model in [4]. 
As identified in Section 2, the most existing MDE tools are developed on the top of EMF – eclipse modeling framework using Ecore-based meta-modeling feature. DL will be applied to the existing MDE tools (e.g. currently, to UML designer, to ThingML which is an EMF- and Sirius-based [10] domain-specific modeling tool [9] and to SsML in future). Specific DSL will be derived from the EMF-based Ecore meta-model [8] based on the standard UML profiles. 
Service Orchestration. In order to embed the collaborative MDE support behind the existing MDE tools for smart city architectures, there is a need for several operational services to perform certain collaborative modeling operations. These operations might, for instance, be calculating modeling deltas by listening for changes or comparing subsequent revisions, applying or propagating modeling deltas on models, etc. The collaborative MDE infrastructure [4] further provides a catalog of supplementary services that can recognize the DL syntax as well as manipulate and reuse DL-based modeling deltas. After generating a specific DL for a domain-specific modeling language, these services can be orchestrated in order to perform certain operations in collaborative MDE of smart city architectures. In the following, these services and their orchestrations are utilized in collaborative modeling support for smart city architectures.
Applications. In the collaborative development of smart city architectures, the overall infrastructure of collaborative MDE has to provide two main scenarios: (1) interactivity of collaborators; (2) consistency of centralized artifact repositories. On the one hand, providing interactivity among collaborators is a main concern. Collaborators may use domain-specific MDE tools for working on their local workspaces. But, there must be a support for joining/opening centralized and shared smart city architectures and working on it in parallel with other colla­bo­ra­tors. Simultaneously, the changes, they make on their copies of model, should be synchronized with other parallel instances in real-time, which is referred as interactivity. On the other hand, smart city architectures and models have to be stored in the centralized and persistent repositories. It allows for storing the histories of architectures and models under development and evolution. Repositories can then store and persist models and their histories safely for further reuse and manipulations. 
The collaborative MDE infrastructure [4] introduces two main scenarios of collaborative MDE namely micro-versioning and macro-versioning. Micro-versioning together with domain-specific MDE tools enables interactivity of several designers and developers (by concurrent collaboration) on the shared and centralized model-driven smart city architectures. Macro-versioning provides the consistency of model-driven smart city architectures. It is a centralized modeling delta repository with model management features such as opening working copies of models, reverting their revisions, storing complete model and their revisions, etc.