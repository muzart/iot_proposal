This section shortly reviews MDE approaches and methodologies for developing smart city applications that the collaborative MDE infrastructure can be applied to.

MDE benefits from its capabilites to model different, but integrated views on systems, which eventually  are made executable for different platforms. The basic idea of the Internet of Things vision is the pervasive presence around us of a variety of things or objects such as sensors, actuators, mobile phones, etc. which are able to interact with each other and cooperate with their pairs to reach common goals \citep{Atzori+2010, Giusto+2010}. This section shortly reviews model-driven approaches and methodologies for developing IoT applications. MDE has a much wider set of modeling techniques and a much more detailed separation of different views and concerns \citep{KateuleWinter2018}. For instance, components have well-defined interfaces and ports.

There are several MDE approaches for developing IoT applications, e.g. the Sirius-based ThingML language \cite{Fleurey+2011}. These model-driven approaches provide very expressive modeling of systems, possibly with code generation. The motivation for model-driven development is to describe a system on a higher level of abstraction. This is usually done in UML and other languages by diagrams modeling specific aspects or views of a system.

MDE techniques are proposed to reduce the severity of IoT applications' development. In such an approach, applications are specied using high-level and abstract mode ls and then given as input to code generators which produce low-level code as output. For instance, PervML \cite{serral2010towards} allows developers to specify pervasive systems at a high-level of abstraction through a set of models (in UML). Nevertheless, such approaches typically require expertise in the modeling language which stakeholders might not be willing to acquire.

Ciccozzi and Spalazzese introduced MDE4IoT \cite{ciccozzi2016mde4iot}, a Model-Driven Engineering Framework supporting the modelling of Things and self-adaptation of Emergent Configurations of connected systems in the IoT. As IoT system consist of several devices, there could emerge some failures in performance of the overall system because of some non-responding devices. The article considers that in such cases the system should adapt to work without the need for these devices, and reinstall and maintain its activities. Moreover, MDE4IoT is meant to exploit the combination of a set of domain-specific modelling languages to achieve separation of concerns.

UML designer \cite{umlDesigner} is another domain-specific modeling framework built on Sirius and EMF. In UML designer, architecture models may describe the logical role of classes by class diagrams, system-actor interactions by use case diagrams, logical component models by component diagrams as well as deployment diagrams, which show the mapping of components to physical entities. Behavior can be described by sequence diagrams, or by state machines and activity diagrams. Activity diagrams describe the actions and object and control flows between actions. State machine diagrams are used in several embedded domains to model the behavior of specific objects e.g. the discrete behavior of components, in a model-driven environment, is usually defined through finite state machines. 

The research presented in \cite{Corredor+2012} takes advantage of the MDE principles to build a holistic development methodology involving a common, semantically expressive abstraction model, to specify a smart space with its specific services. It proposes the Resource-Oriented and Ontology-Driven Development (ROOD) methodology, which improves traditional MDE-based tools through semantic technologies for rapid prototyping of smart spaces according to the IoT paradigm. In the framework of ROOD, the Smart Space Modeling Language (SsML) was developed based on UML, that defines a Domain Specific Model (DSL). It can be used for describing high-level behaviors, interactions and context information of the entire smart space. It further defines the processing aspects related to the sensing and actuating capabilities of the smart objects, as well as the context model they manage; moreover, encapsulate these concepts into RESTful resources. The ROOD approach is realized using Obeo Designer \cite{Designer2016}.

Patel et. al presented a multi-stage model driven approach for IoT application development, based on an identification of the skills and responsibilities of the various stakeholders involved in the process \cite{patel2015enabling}.  Noteworthy in their approach is the use of customizable modeling languages that are customized for a particular stakeholder task and application area, where abstractions available to one stakeholder are generated from information provided by other stakeholders at previous stages. The approach is complemented by methods for generating code and mapping tasks that lead to the deployment of node-level code on composite devices.