MDE benefits from its capabilites to model different, but integrated views on systems, which eventually  are made executable for different platforms. The basic idea of the Internet of Things vision is the pervasive presence around us of a variety of things or objects such as sensors, actuators, mobile phones, etc. which are able to interact with each other and cooperate with their pairs to reach common goals [1][5]. This section shortly reviews model-driven approaches and methodologies for developing IoT applications. MDE has a much wider set of modeling techniques and a much more detailed separation of different views and concerns [7]. For instance, components have well-defined interfaces and ports.

There are several MDE approaches for developing IoT applications, e.g. the Sirius-based ThingML language [6]. These model-driven approaches provide very expressive modeling of systems, possibly with code generation. The motivation for model-driven development is to describe a system on a higher level of abstraction. This is usually done in UML and other languages by diagrams modeling specific aspects or views of a system.

UML designer [9] is another domain-specific modeling framework built on Sirius and EMF. In UML designer, architecture models may describe the logical role of classes by class diagrams, system-actor interactions by use case diagrams, logical component models by component diagrams as well as deployment diagrams, which show the mapping of components to physical entities. Behavior can be described by sequence diagrams, or by state machines and activity diagrams. Activity diagrams describe the actions and object and control flows between actions. State machine diagrams are used in several embedded domains to model the behavior of specific objects e.g. the discrete behavior of components, in a model-driven environment, is usually defined through finite state machines. 

The research presented in [12] takes advantage of the MDE principles to build a holistic development methodology involving a common, semantically expressive abstraction model, to specify a smart space with its specific services. It proposes the Resource-Oriented and Ontology-Driven Development (ROOD) methodology, which improves traditional MDE-based tools through semantic technologies for rapid prototyping of smart spaces according to the IoT paradigm. In the framework of ROOD, the Smart Space Modeling Language (SsML) was developed based on UML, that defines a Domain Specific Model (DSL). It can be used for describing high-level behaviors, interactions and context information of the entire smart space. It further defines the processing aspects related to the sensing and actuating capabilities of the smart objects, as well as the context model they manage; moreover, encapsulate these concepts into RESTful resources. The ROOD approach is realized using Obeo Designer [13].