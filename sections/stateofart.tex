There are several MDE approaches for developing IoT applications, e.g. the Sirius-based ThingML language \cite{Fleurey+2011+1}. These model-driven approaches provide very expressive modeling of the IoT-based smart architectures, possibly with code generation. The motivation for model-driven development is to describe a system on a higher level of abstraction. This is usually done in UML and other languages by diagrams modeling specific aspects or views of model-driven architectures for smart systems. This section shortly reviews these existing model-driven approaches and methodologies for developing IoT applications using MDE.

MDE techniques are proposed to reduce the severity of IoT applications' development. In such an approach, applications are specified using high-level abstractions using models. These models are then used to produce deployable source code. For instance, PervML \cite{serral2010towards} enables developers to specify their software architectures at abstraction levels through a set of models (in UML).

Ciccozzi and Spalazzese introduced MDE4IoT \cite{ciccozzi2016mde4iot}, a Model-Driven Engineering Framework supporting the modeling of Things and self-adaptation of Emergent Configurations of connected systems in the IoT-based smart systems. As the IoT systems consist of several connected software services and hardware components, there might be possible failures in performance of the overall system because of some non-responding devices. The article considers that in such cases the system should adapt to work and sustain without a need for these inactive devices, and re-install and maintain its activities. In order to avoid such failures, MDE4IoT is meant to exploit the combination of a set of domain-specific modeling languages to achieve separation of concerns.

The research presented in \cite{Corredor+2012} uses the MDE principles to build a holistic development methodology involving a common, semantically expressive abstraction model, to specify a smart space with its specific services. It proposes the Resource-Oriented and Ontology-Driven Development (ROOD) methodology, which improves traditional MDE-based tools through semantic technologies for rapid prototyping of smart spaces according to the IoT paradigm. In the framework of ROOD, the Smart Space Modeling Language (SsML) was developed based on UML, that defines a Domain Specific Model (DSL). It can be used for describing high-level behaviors, interactions and context information of the entire smart space. It further defines the processing aspects related to the sensing and actuating capabilities of the smart objects, as well as the context model they manage; moreover, encapsulate these concepts into RESTful resources. The ROOD approach is realized using Obeo Designer \cite{Designer2016}.

Patel et. al. \cite{patel2015enabling} presents a multi-stage model-driven approach for IoT application development, based on identification of the skills and responsibilities of the various stakeholders involved in the process. The approach uses configurable modeling languages that are customized for a particular stakeholder task and application area, where abstractions available to a specific stakeholder are generated from information provided by other stakeholders at previous stages. The approach is complemented by methods for generating code and mapping tasks that lead to the deployment of node-level code on composite devices.

ThingML \cite{Fleurey+2011} is another domain-specific modeling framework built on Sirius. In ThingML, the state machine diagrams are used in several embedded domains to model the behavior of specific objects e.g. the discrete behavior of components. In the MDE paradigm of ThingML, the states of hardware components are managed by defining finite state machines.