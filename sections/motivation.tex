Realization of smart systems requires to understand the complex processes, so that proper decisions can be made based on available data. It is important to manage large and distributed networks of devices. These devices need to communicate, they need to be self-stable against errors and possible failures. In addition, they produce a lot of data that needs to be analyzed and stored. This requires to consider all aspects like above in modeling.

All challenges of smart distributed systems can be more or less eased by generic reference architectures \cite{Fleurey+2011}, \cite{Gubbi+2013}, \cite{KateuleWinter2018}. From a technical point of view, all IoT-based smart systems can be treated as a large, sensor- and actor-based, distributed information system. These systems can then be generalized as a large, sensor-based software reference architecture \cite{KateuleWinter2018} which include main activities like data collection by sensors, data processing by servers, control by activators and data support for actors.

MDE provides a collection of modeling concepts and the detailed separation of different views and concerns \cite{KateuleWinter2018}. For instance, components have well-defined interfaces and ports. MDE benefits from its capabilities to model different, but integrated views and behavior of IoT systems which are eventually made executable for different smart platforms. The basic idea of the smart system vision is the pervasive presence of a variety of things or objects such as sensors, actuators, mobile phones, etc. which are able to interact with each other and cooperate with their pairs to reach common goals \cite{Atzori+2010} \cite{WeberRomana2010}.