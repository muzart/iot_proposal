A smart city promises not only a more efficient management of resources, but also an increase in the quality of services provided, while still remaining cost-effective. It is very important to keep the citizens as central stakeholders in mind, which is why they must be included in all stages of Smart City development, starting with the planning stage. Smart cities are more flexible than traditional city management approaches in case of unexpected events.

Implementing a smart city requires to understand the complex processes in cities, so that proper conclusions can be drawn from available data. Depending on the specifics, this can be achieved either through human analysis or through artificial intelligence, periodically or in real time. It is also necessary to manage large, distributed and diverse networks of devices. These devices need to communicate, they need a reliable power supply, and they need to be resilient against errors and sabotage. In addition, they produce enormous amounts of real-time data that needs to be analyzed and stored. This requires expertise in many different areas of research and the ability to work on interdisciplinary projects.

All aforementioned challenges of smart distributed systems can be more or less eased by generic reference architectures \cite{Fleurey+2011}, \cite{Gubbi+2013}, \cite{KateuleWinter2018}. From a technical point of view, all IoT-based smart systems can be treated as a large, sensor- and actor-based, distributed information system. These systems can then be generalized as a large, sensor-based software reference architecture \cite{KateuleWinter2018} which include main activities like data collection by sensors, data processing by servers, control by activators and data support for actors.

MDE provides a collection of modeling concepts and the detailed separation of different views and concerns \cite{KateuleWinter2018}. For instance, components have well-defined interfaces and ports. MDE benefits from its capabilities to model different, but integrated views and behavior of IoT systems which are eventually made executable for different smart platforms. The basic idea of the smart city vision is the pervasive presence of a variety of things or objects such as sensors, actuators, mobile phones, etc. which are able to interact with each other and cooperate with their pairs to reach common goals \cite{Atzori+2010} \cite{WeberRomana2010}.