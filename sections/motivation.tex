A smart city promises not only a more efficient management of resources, but also an increase in the quality of services provided, while still remaining cost-effective. It is very important to keep the citizens as central stakeholders in mind, which is why they must be included in all stages of Smart City development, starting with the planning stage. Smart cities are more flexible than traditional city management approaches in case of unexpected events.

From a technical point of view, a smart city can be seen as a large, sensor- and actor-based, distributed information system. It is large because it scales to city-size, it is sensor-based because sensors provide the data required to fuel the system, and it is distributed because its sensors need to span the entire area. It is an information system because the collected data is stored, analyzed, and used to start and control city processes and application.

Implementing a smart city requires to understand the complex processes in cities, so that proper conclusions can be drawn from available data. Depending on the specifics, this can be achieved either through human analysis or through artificial intelligence, periodically or in real time. It is also necessary to manage large, distributed and diverse networks of devices. These devices need to communicate, they need a reliable power supply, and they need to be resilient against errors and sabotage. In addition, they produce enormous amounts of real-time data that needs to be analyzed and stored.

This requires expertise in many different areas of research and the ability to work on interdisciplinary projects.