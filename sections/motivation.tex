Bugungi kunda Smart city konsepsiyasi o’z ichiga bir qancha komponentlarni oladi. Ularga misol qilib Smart Grid, Smart traffic control, Smart Homes, Smart Cars, Smart factories va boshqalarni aytsa bo’ladi. Bu turdagi elementlarning hammasi umumiy qilib “Smart X” deb nomlanadi. 
Smart city konsepsiyasi shaharning axborot texnologiyalardan foydalangan holda boshqarish, avtomatlashtirish, optimallashtirishga muxtoj barcha sohalarini hisobga olgan holda kelib chiqadi. 
Demak, Smart city turli Smart X komponentlaridan tashkil topganligi bois, uni ishlab chiqish murakkab va kompleks jarayondir va uni umumiy modellashtirishda turli texnik va jamoaviy soha ekspert va mutaxassislarining birgalikda hamkorlikda ishlashini talab etadi. Masalan, Smart cityda sensorlar va aktuatorlarni loyihalash embedded system mutaxassislari tomonidan amalga oshirilishi lozim. Ushbu sensor va turli qurilmalardan keladigan ma’lumotlarni qabul qilish, yig’ish va saqlash tizim administratorlari (texnik mutaxassislar) tomonidan amalga oshirilsa, ushbu yi’g’ilgan ma’lumotlarni tahlil qilish va qayta ishlash analitiklar tomonidan amalga oshirilishi maqsadga muvofiq. IoT qurilmalari, sensorlari va aktuatorlarini to’g’ri joylashtirish masalasi ham muhim hisoblanib, u bilan qurilish loyihalashtiruvchilari shug’ullanishi lozim. Dasturiy injinerlar ushbu qurilmalarning o’zaro aloqasini ta’minlovchi va turli xizmatlar ko’rsatishi uchun mos dasturiy vositalarni ishlab chiqishadi. Bundan tashqari boshqa soha vakillari ham ushbu jarayonda ishtirok etadi. Bu mutaxassislarning birgalikda loyihalash jarayonlarini tashkillashtirish muhim ahamiyat kasb etadi. 

Bu jarayonning yakuniy natijasi sifatida katta arxitektura hosil bo’ladi. Bu arxitekturani saqlash va boshqarish uchun mos injiniring metodologiyalarini qo’llash talab etiladi.  Model-driven engineering (MDE) shu kabi ishlarni bajarish uchun mos fundament sifatida xizmat qilishi mumkin. Model-driven yondashuvlarini qo’llagan holda intensive hamkorlikda ishlash o’ziga xos instrument va texnikalarni talab etadi. 

Ushbu jarayonda qurilgan modellarni saqlashning ham ahamiyati katta. Model yaratilishidagi bajarilgan ketma-ketliklar tarixi ham muhim ahamiyat kasb etadi. Chunki, modellashtirishda ham ma’lum vaqtdan keyin ma’lum bir muddat oldingi holatga qaytish talab etiladigan vaziyatlar bo’ladi.

Bu jarayonlarning hammasi hozirgi kunda dasturiy ta’minotlarni ishlab chiqishda qo’llaniladigan jamoaviy dasturlash metodologiyalariga o’xshab ketadi. Ya’ni hozirda dasturchilar SVN, Git, Mercurial kabi tizimlarda proekt manager, koder, dizayner, testerlar birgalikda onlayn tarzida biror bir sohani dasturiy ta’minotini ishlab chiqishadi. Bunda ular o’ziga tegishli vazifani bajargan holda, uni umumiy repozitoriyaga yuklashadi. Umumiy qilingan ish repozitoriyada saqlanib boriladi.
Demak, Smart X turkumidagi Smart city loyihalarini modellashtirishda ham MDE va jamoaviy dasturlash metodologiyalarini qo’llash yaxshi samara berishi mumkin. But, the current state of collaborative MDE is still a long way from realizing its full potential, and the adoption in industry and IoT remains limited.
