In the course of this research, several benefits are expected to be achieved. This research starts by investigating the existing MDE approaches for IoT-based smart systems and collaboration support for them. The state of the art will be studied and documented for further research. The core research aims to bring several advantages to the field of model-driven IoT with collaborative modeling support.
\begin{itemize}
\item \textbf{Interactivity:} Large-scale model-driven architectures for IoT systems are developed by the teams of developers. These develops are distributed over long distances. For incremental and agile model-driven IoT development and engineering there is a need for truly collaborative work. By the real-time synchronization of change states, interactivity of developers can achieved which sees the centralized model-driven architectures as a single point of truth. The synchronization of the changes will be facilitated by the exchange of small DL-based modeling deltas \cite{Kuryazov+2018}. Various model design tools (e.g., PervML, ThingML, SsML) can interact with each other in terms of DL.
\item \textbf{Consistency:} During the evolutionary life-cycle, model-driven IoT architectures result in several revisions that must be consisted and maintained for long-term use. With collaborative development of model-driven IoT architectures, the histories of evolving architectures are consistently preserved in repositories by macro-versioning for further reuse and analysis.
\end{itemize}

The most importantly, DL will be a common underlying language for supporting both interactivity and consistency.

While doing research, the results of research activities will be published and discussed among scientists and researchers in the field.