In the course of this research, several benefits are expected to be achieved. This research starts by investigating the existing MDE approaches for IoT-based smart systems. The state of the art will be studied and documented for further research. The core research aims to bring several advantages to the field of model-driven IoT.
\begin{itemize}
\item \textbf{Adaptation:} Nowadays, MDE promises high-level model-driven technologies and concepts like UML. But, as long as IoT is a novel domain, the current potential of MDE is not fully utilized in developing IoT systems. As the result of this research, IoT systems benefit from the well-developed MDE concepts and technologies. To this end, standard UML profies will be extended and adapted to meet the requirements of IoT-based system development. In the resulting domain-specific meta-model, the standard UML profiles will be enriched with the domain-specific aspects of IoT-based smart systems.
\item \textbf{Domain-Specific MDE Tool:} After developing a domain-specific meta-model for IoT, a domain-specific MDE tool will be developed based on that meta-model. The resulting domain-specific tool helps to design different perspective of IoT systems resulting in model-driven IoT systems. 
\end{itemize}

While doing research, the results of research activities will be published and discussed among scientists and researchers in the field.