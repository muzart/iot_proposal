The central theme in this paper is to contribute towards a needs-based improvement of MDE techniques, methods, and tools, and researching novel, smart modeling techniques and applications, to address challenges posed by future software-intensive systems in the framework of IoT. Some of the most important challenges include the following:

\textbf{Interactivity:} The interconnected world enables global software engineering, with developer teams being distributed over long distances already becoming a common practice. For incremental and agile model-driven development and engineering, this highlights a need for truly model-based collaborative work, and tools for collaborative modeling. The micro-versioning is to be considered as a main foundation for providing interactivity support treating a centralized model as a single point of truth. Synchronization of changes will be eased by exchanging small DL-based modeling deltas. The various model designing tools (e.g. UML Designer, ThingML, SsML, etc.) which are running on different platforms can communicate with each other in terms of DL. Thereby, DL-based collaborative modeling environment serves as a common underlying infrastructure for all. 

\textbf{Consistency:} Both the heterogeneity and the high complexity of cyber-physical systems make integrated views and models of diverse development artifacts and their interrelationships indispensable. Model consistency and integration needs to occur both throughout evolutionary life-cycle of the systems/models under development. With collaborative modeling, the histories of evolving software models are consistently preserved in modeling delta repositories by macro-versioning for further reuse and analysis. The DL syntax fully satisfies MDE concepts and provides consistency of modeling deltas as well as models under development and evolution.

\textbf{Flexibility:} IoT applications are dynamic, fast-evolving, and based on heterogeneous parts, e.g. realized as micro-services. This raises a need for a technology-agnostic approaches to integrate diverse subsystems flexibly. MDE can be applied for model-driven systems integration, bridging the gap between service and implementation layers using model transformations, while preserving separation of concerns. DL does not rely on any specific underlying implementation or technical space being fully independent underlying problem domain. The DL-based modeling deltas may form MDE artifacts confirming to a given meta-model, yet consist of whole system or represent changes. DL services are also realized based on service-oriented paradigm, the underlying DL and its services can easily be adapted and extended according to new changes or requirements which arise from the IoT application domain. 