The main topic of this article is to contribute to the improvement of methods, techniques and tools MDE on the basis of need, as well as the exploration of new, intelligent modeling techniques and applications to address the challenges associated with the future of software-intensive systems in the IoT. Some of the most important problems include:

\textbf{Interactivity:} An interconnected world provides global software development, while development teams are distributed over long distances, which are already becoming common practice. For incremental and agile model-driven development and engineering there is a need for truly model-based collaborative work, and tools for collaborative modelling. Microversion should be considered as the main basis for providing interactive support, which views the centralized model as a single point of truth. The synchronization of the changes will be facilitated by the exchange of small DL-based modeling deltas. Various model design tools (for example, UML Designer, ThingML, SsML, etc.), which work on different platforms, can interact with each other in terms of DL. Thus, DL-based collaborative modeling environment serves as a common base for all. 

\textbf{Consistency:} Both heterogeneity and high complexity of cyberphysical systems make integrated views and models of various artifacts of development and their interrelations irreplaceable. The consistency and integration of the model must occur both during the evolutionary life cycle of the systems / models being developed. With collaborative modeling, the histories of evolving software models are consistently preserved in modeling delta repositories by macro-versioning for further reuse and analysis. The DL syntax fully satisfies MDE concepts and ensures the consistency of modeling deltas, as well as models under development and evolution.

\textbf{Flexibility:} IoT applications are dynamic, fast-evolving and are based on heterogeneous parts, for example. implemented as micro-services. This creates the need for technological agnostic approaches for the flexible integration of various subsystems. MDE can be used to integrate model-based systems, eliminating the gap between service levels and implementation using model transformations while preserving the separation of concerns. DL does not rely on any particular underlying implementation or technical space, which is a completely independent basic problem domain. DL-based modeling deltas can create MDE artifacts that support this metamodel, but consist of a whole system or represent changes. DL services are also implemented on the basis of a service-oriented paradigm, the underlying DL and its services can be easily adapted and expanded in accordance with new changes or requirements that arise from the domain of the IoT application.