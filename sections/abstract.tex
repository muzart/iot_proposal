\textbf{Abstract.} Smart technologies employ sensors, actors, communication protocols and home servers to increase the efficiency of city services and processes. Information and Communication Technology (ICT) helps optimizing the processes, while the Internet of Things (IoT) provides the platform for managing a multitude of small sensor and actor devices for smart technologies. Smart technologies promises not only a more efficient management of resources, but also an increase in the quality of services provided, while still remaining cost-effective. 

From a technical point of view, "Smart X" systems can be seen as a large, sensor-based, distributed information system with data collection, data processing and data support components. Nowadays, smart systems and software architectures are reaching new levels of complexity, necessitating appropriate engineering methodologies. Model-Driven Software Engi­neer­ing provides the required foundations for formally define generic architectures and development of software support on the top of IoT. To cope with ever growing complexity of such software architectures, model-driven approaches can extensively be applied to modeling of the IoT technologies as well as developing and maintaining software systems and platforms for "Smart X" technologies. As the models of these architectures and platforms become huge and complex over time, model-driven development is necessary to cope with development and evolution challenges arising from the IoT applications. This research proposal aims at (1) studying the state of the art in model-driven IoT, (2) extended research in model-driven IoT focusing on the "Smart X", (3) model-driven IoT for developing software architectures and platforms for IoT-based systems. As the proof of concepts, it further focuses on applying model-driven engineering, new upcoming trends in model-driven IoT, and its adoption in developing smart technology architectures.