There are several domain-specific MDE approaches and tools for developing IoT-based architectures and applications. However, research in model-driven IoT is in its early stages and extended research in the field is required. 

As explained in Section~\ref{sec:stateofart}, there are already several domain-specific MDE notations and tools that can be used for designing, modeling and developing IoT-based smart system architectures and applications. They can be distinguished by different design aspects and perspectives such as views, viewpoints, components, communication protocols, etc. However, MDE is not fully utilized for developing model-driven IoT approaches.

The core objectives of this proposal are threefold:
\begin{itemize}
\item[--] \textbf{Studying The State of Art.} This research initially indends to study the state of the art in model-driven IoT approaches in order to identify the MDE tools and approaches \cite{Fleurey+2011}, \cite{Corredor+2012} dedicated especially to develop the IoT-based smart systems. The most existing MDE approaches and tools for IoT-based software systems employ standard UML profiles developed on the top of EMF \cite{Steinberg+2008}. As long as these tools are open-source and their underlying UML meta-models of these approaches can be further developed and extended. 
\item[--] \textbf{Research.} The most important, challenging and long-running part of this research focuses on extended research in model-driven IoT focusing on the smart technologies. This research aims to utilize the full potential of MDE in developing advanced model-driven IoT systems, applications and architectures.
\item[--] \textbf{Validation.} This research further focuses on applying a proposed approach to real-world applications as the proof of concept.
\end{itemize}

 