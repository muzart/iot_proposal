As explained so far, there are already several domain-specific MDE notations and tools that can be reused for designing, modeling and developing IoT-based smart system architectures and applications. They can be distinguished by different design aspects and perspectives such as views, viewpoints, components, communication protocols, etc. For the sake of interactivity by collaborative development, maintenance and consistency, evolution and flexibility, they mostly lack collaborative development support by sharing the artifacts of smart system architectures, models and applications among collaborators. 

The core research objectives of this proposal are threefold:
\begin{itemize}
\item[--] \textbf{Studying The State of Art.} This research initially indends to study the state of the art in model-driven IoT approaches in order to identify the best domain-specific MDE tool candidates \cite{Fleurey+2011}, \cite{Corredor+2012} dedicated especially to develop the IoT-based smart systems. The most existing MDE approaches and tools for IoT-based software systems employ standard UML profiles developed on the top of EMF \cite{Steinberg+2008}. As long as these tools are open-source and their underlying UML meta-models of these approaches can be reused and extended for collaborative modeling and architecture development for the IoT-based smart systems. Furthermore, this research reviews the existing collaborative MDE approaches for IoT systems in order to identify research challenges in the field. 
\item[--] \textbf{Research.} The most important, challenging and long-running part of this research focuses on extended research in collaborative MDE approaches for IoT focusing on the smart technologies. In this phase, the existing collaborative modeling approaches will be studied making a operation-based textual difference representation language \cite{Kuryazov+2018} as a potential candidate. According to \cite{Kuryazov+2018}, difference representation language is considered to be generic and as one of the efficient approach for collaborative modeling covering all aspects of collaborative modeling.
\item[--] \textbf{Validation.} This research further focuses on applying a proposed approach to real-world applications as the proof of concept.
\end{itemize}

 