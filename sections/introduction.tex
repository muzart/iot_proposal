The internet of the present day is enriched with a huge amount of various devices and micro services. These devices and services operate for different purposes, yet work together to achieve common goals. The various types of sensors, actuators and devices are being developed to provide a range of services. These sensors and devices have created a new technological trend today. This trend has been named the "Internet of Things" \cite{WeberRomana2010}, "Network of Things" \cite{Gubbi+2013} or "Web of Things" \cite{Guinard+2010}. 

In the framework of IoT, more and more devices are equipped with network connectivity to autonomously provide "smarter" services, forming the Internet of Things (IoT). Applications are wide-ranging, and have variously been termed "Smart X", including Smart Homes, Smart Factories (Industry 4.0), Smart Government, Smart City, Smart Grid, Smart Traffic Control, and many more \cite{Kuryazov+2018+1}.

Smart technologies employ sensors and actors to increase the efficiency of services and processes, including environmental sustainability, energy efficiency, mobility, health care, safety, and security. ICT helps to optimize the processes, while IoT provides the platform for managing a multitude of small sensor, actor devices, communication protocols and home servers.

IoT \cite{WeberRomana2010} is the basic concept for developing the network of things equipped with built-in technologies for interaction with each other or with the environment. The concept was formulated in 1999 as an understanding of the prospects for the widespread use of radio frequency identification means for the interaction of physical objects with each other and with the external environment. 
\begin{definition}
\textbf{Internet of Things}\label{definition:internetOfThings}\\
The Internet of Things is about installing sensors (RFID, IR, GPS, laser scanners, etc.) for everything, and connecting them to the internet through specific protocols for information exchange and communications, in order to achieve intelligent recognition, location, tracking, monitoring and management \cite{Atzori+2010}.
\end{definition}

The concept of smart technologies arises from the need to manage, automate, optimize and explore all aspects of daily life that could be improved and optimized by information technologies. The software paradigm IoT, being a core concept behind smart smart technologies, is largely perceived as a collection of interconnected "things" within smart technologies.

The IoT-based smart applications are realized by interconnected systems of heterogeneous hardware, software, and embedded systems: these cyber-physical systems introduce new levels of complexity, requiring appropriate engineering methodologies to support formally rigorous software and systems development \cite{Kuryazov+2018+1}. \textbf{Model-Driven Engineering (MDE)} provides fitting foundations and is considered as an enabling technology for advancing smart technology applications.

A collection of software and hardware components for smart systems can be viewed as a huge reference architecture based on model-driven software platform. Thus, a number of model-driven reference architectures \cite{KateuleWinter2018} and models \cite{Yin+2015} are introduced for smart system development. These architectures and models contemplate smart system architectures as a blueprint which provides an appropriate level of abstraction for the development process of smart systems. Model-driven reference architectures are used to represent and define different development aspects of smart system architectures consisting of different views, viewpoints, software services and components like home servers, communication protocols, sensors, activators, etc., in a single, huge architecture. For instance, several model-driven approaches are investigated in \cite{Corredor+2012} and \cite{KleanthisFoivos2016} utilizing different modeling language profiles (e.g. standard UML profiles \cite{Rumbaugh+2004}) in development of smart system architectures.

MDE is the modern day approach of software system development which supports well-suited abstraction concepts for development activities. It intends to improve the productivity of the design and development, maintenance activities, and communication among various actors and stakeholders of a system. As the main concept in MDE, models are well-suited for designing, developing and producing large-scale software projects. MDE brings several main benefits such as a productivity boost, models become a single point of truth \cite{Fleurey+2011}. Models are the main artifacts in MDE. They are well-suited for designing, developing and producing large software systems. Software models are the documentation and implementation of software systems \cite{Kleppe+2003}.

This proposal focuses on the field of model-driven IoT for smart systems. Its objectives are manifold: (1) studying the state of the art in model-driven IoT, (2) extended research in model-driven IoT focusing on smart technologies, (3) applying MDE concepts to develop software architectures and platforms for IoT-based systems.

This research proposal is structured as follows: Section~\ref{sec:motivation} motivates the research field defining its contributions and novelty. Section~\ref{sec:stateofart} investigates existing model-driven IoT approaches. The main research objectives of this proposal are sketched in Section~\ref{sec:researchObjectives}. Section~\ref{sec:expectedBenefits} discusses the main expected benefits of the research. This paper ends up in Section~\ref{sec:conclusion} with some conclusions.