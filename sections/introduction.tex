
The Internet of the present day is enriched with various devices. These devices serve different purposes. Various types of sensors, actuators and devices are being developed to provide a range of services. These sensors and devices have created a new technological trend today. This trend has been named the "Internet of Things" or "Network of Things". 

\textbf{Internet of Things (IoT)} is a concept of a network of things equipped with built-in technologies for interaction with each other or with the environment. The concept was formulated in 1999 as an understanding of the prospects for the widespread use of radio frequency identification means for the interaction of physical objects with each other and with the external environment. The Internet of Things is about installing sensors (RFID, IR, GPS, laser scanners, etc.) for everything, and connecting them to the internet through specific protocols for information exchange and communications, in order to achieve intelligent recognition, location, tracking, monitoring and management [1]
[1] (https://www.sciencedirect.com/science/article/pii/S0167739X17305253). 

IoT konsepsiyasining rivojlanishi, unga bog’liq bo’lgan yangi tadqiqot yo’nalishlariga yo’l ochdi. Shu konsepsiyalardan biri Smart city konsepsiyasi hisoblanadi. 

\textbf{Smart City} - the concept of integrating several information and communication technologies (ICT) and the Internet of things (IoT) solutions for the management of urban property. The purpose of creating a "smart city" is to improve the quality of life with the help of computer technologies to improve the efficiency of services and meet the needs of people. ICT allows the city government to directly interact with communities and urban infrastructure, and monitor what is happening in the city, how the city is developing, and what can improve the quality of life. Through the use of sensors integrated in real time, the accumulated data from urban residents and devices are processed and analyzed. 
