The internet of the present day is enriched with a huge amount of various devices and micro services. These devices and services operate for different purposes, yet work together to achieve common goals. Various types of sensors, actuators and devices are being developed to provide a range of services. These sensors and devices have created a new technological trend today. This trend has been named the "Internet of Things", "Network of Things" or "Web of Things" [reference]. 

\textbf{Internet of Things (IoT)} [reference] is the basic concept for developing the network of things equipped with built-in technologies for interaction with each other or with the environment. The concept was formulated in 1999 as an understanding of the prospects for the widespread use of radio frequency identification means for the interaction of physical objects with each other and with the external environment. 
\begin{definition}
\textbf{Internet of Things}\label{definition:internetOfThings}\\
The Internet of Things is about installing sensors (RFID, IR, GPS, laser scanners, etc.) for everything, and connecting them to the internet through specific protocols for information exchange and communications, in order to achieve intelligent recognition, location, tracking, monitoring and management [reference].
\end{definition}

In framework of IoT, more and more devices are equipped with network connectivity to autonomously provide "smarter" services, forming the Internet of Things (IoT) [reference]. Applications are wide-ranging, and have variously been termed "Smart X", including Smart Homes, Smart Factories (Industry 4.0), Smart Government, Smart City, Smart Grid, Smart Traffic Control, and many more.

\textbf{Smart City}. The concept of integrating several information and communication technologies (ICT) and the Internet of things (IoT) solutions for the management of urban property. The purpose of creating a "smart city" is to improve the quality of life with the help of computer technologies to improve the efficiency of services and meet the needs of people. ICT allows the city government to directly interact with communities and urban infrastructure, and monitor what is happening in the city, how the city is developing, and what can improve the quality of life. Through the use of sensors integrated in real time, the accumulated data from urban residents and devices are processed and analyzed. 

The concept of smart city arises from the need to manage, automate, optimize and explore all aspects of a city that could be improved and optimized by information technologies. The software paradigm IoT, being a core concept behind smart cities, is largely perceived as a collection of interconnected "things" within smart cities. 

\textbf{Model-Driven Engineering.} The IoT-based smart city applications are realized by interconnected systems of heterogeneous hardware, software, and embedded systems: these cyber-physical systems introduce new levels of complexity, requiring appropriate engineering methodologies to support formally rigorous software and systems development. Model-Driven Engineering (MDE) provides fitting foundations and is considered as an enabling technology for advancing Smart X applications. Smart City development is an extensive and complex endeavor, which requires intensive collaboration of various stakeholders and experts from different technical and social domains. Intensive collaboration by using model-driven approaches require broad support by appropriate tools and techniques. Yet, the current state of collaborative MDE is still a long way from realizing its full potential, and the adoption in industry and IoT remains limited.

As a software engineering paradigm, MDE is the modern day style of software and system development which supports well-suited abstraction concepts to development activities. Model-driven engineering intends to improve the productivity of the design and develop­ment, maintenance activities, and communication among various actors and stakeholders of a system. In MDE, software models (e.g. in UML [reference]) which also comprise source code are the central key artifacts. They are well-suited for designing, developing and producing large-scale software projects. Software models are the documentation and implementation of software systems being developed and evolved [reference]. MDE brings several main benefits such as a productivity boost and , models become a single point of truth., and theyModels are reusable and automatically kept in sync with the code they represent [reference]. Software models (e.g. in UML [reference]) are the key artifacts in MDE activities. They are well-suited for designing, developing and producing large-scale software projects. Software models are the documentation and implementation of software systems being developed and evolved [reference].

Models are constantly changed during their development and evolutionary life-cycle by various developers and experts. They are constantly evolved and maintained undergoing diverse changes such as extensions, corrections, optimization, adaptations and other improvements. All development and maintenance activities contribute to the evolution of software models resulting in several subsequent revisions. During software evolution, models become large and complex raising a need for collaboration of several developers, designers and stakeholders (i.e. collaborators) on shared models i.e. \textit{Collaborative modeling}.

A meta-model generic, modeling tool generic and flexible collaborative modeling infrastructure was introduced in [4]. The associated technical support also focuses on providing a catalog of supplementary services which are underlying services for developing collaborative modeling applications.

This research proposal is structured as follows: Section~\ref{sec:motivation} motivates the research field defining its contributions and novelty. Section~\ref{sec:stateofart} investigates existing model-driven IoT approaches and collaborative modeling approaches for IoT. The main research objectives of this proposal are sketched in Section~\ref{sec:researchObjectives}. This paper ends up in Section~\ref{sec:conclusion} by drawing some conclusions.